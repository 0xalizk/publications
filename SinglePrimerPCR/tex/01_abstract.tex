%begin_custom_header
%begin_custom_header
\documentclass[11pt]{article}	% RECOMB: "at least 11 point font size on U.S. standard 8 1/2 by 11 inch paper with no less than one inch margin all around."				
\usepackage[utf8]{inputenc}   % umlauts etc.
\usepackage[english]{babel}
\usepackage [autostyle, english = american]{csquotes}
\MakeOuterQuote{"}
\usepackage{hyperref}
\usepackage{array}
% ----------------------------------
\usepackage[backend=biber,style=nature,sorting=none,url=false]{biblatex}
% url = false. There are also isbn, doi etc., similar options. 
\addbibresource{/Users/mohammedalshamrani/Downloads/School/Waldispul/Publishing/z-misc/zotero-library/my_library.bib}
% ----------------------------------
% Citation style 	biblatex stylename
% ----------------------------------
% 	ACS				chem-acs
% 	AIP				phys (*)
% 	Natur			nature
% 	Science			science
% 	IEEE			ieee
% 	Chicago			chicago-authordate
% 	MLA				mla
% 	APA				apa
% ----------------------------------
% sorting options:
% ----------------------------------
%	nty 		Sort by name, title, year.
%	nyt 		Sort by name, year, title.
%	nyvt 		Sort by name, year, volume, title.
%	anyt 		Sort by alphabetic label, name, year, title.
%	anyvt 		Sort by alphabetic label, name, year, volume, title.
%	ynt 		Sort by year, name, title.
%	ydnt 		Sort by year (descending), name, title.
%	none 		Do not sort at all. All entries are processed in citation order.
% ----------------------------------
\newcommand{\harpoon}{\overset{\rightharpoonup}}
\newtheorem{theorem}{Theorem}
\usepackage{verbatim} % multiline comment
\usepackage{graphicx}
\graphicspath{{/Users/mohammedalshamrani/Downloads/School/Waldispul/Publishing/Paper_04/fig/}}
\setlength\fboxsep{0pt} % figure border padding
\setlength\fboxrule{1pt} % figure outline
\usepackage[fleqn]{amsmath}  % also \documentclass[fleqn]{article}
\usepackage[margin=1in]{geometry}
\abovedisplayskip=0pt
\abovedisplayshortskip=0pt
\belowdisplayskip=0pt
\belowdisplayshortskip=0pt
\setlength{\mathindent}{0pt}
\usepackage{amsfonts} % for R (real numbers)
\usepackage{float}
\usepackage[font=scriptsize,labelfont=bf]{caption}

\usepackage[percent]{overpic}
\usepackage[export]{adjustbox}
% ----------------------------------
%Squeezing the Vertical White Space
%http://www.terminally-incoherent.com/blog/2007/09/19/latex-squeezing-the-vertical-white-space/
% 	THIS FIXES THE PROBLEM OF SUBSECTIONS STARTING IN A NEW PAGE
\setlength{\parskip}{0pt}
\setlength{\parsep}{10pt}
\setlength{\headsep}{0pt}
\setlength{\topskip}{0pt}
\setlength{\topmargin}{0pt}
\setlength{\topsep}{0pt}
\setlength{\partopsep}{10pt}
\usepackage[compact]{titlesec}
\titlespacing{\section}{0pt}{*2}{*2} % {left margin} {above-skip} {below-kip} , The * notation replaces the formal notation using plus/minus and etc. 
\titlespacing{\subsection}{0pt}{*1}{*1}
\titlespacing{\subsubsection}{0pt}{*1}{*1}
% ----------------------------------
\newenvironment{absolutelynopagebreak}
  {\par\nobreak\vfil\penalty0\vfilneg
   \vtop\bgroup}
  {\par\xdef\tpd{\the\prevdepth}\egroup
   \prevdepth=\tpd}
% ----------------------------------
\newcommand{\bfl}{\begin{flushleft}}
\newcommand{\efl}{\end{flushleft}}
\newcommand{\mys }{\hspace{0.1cm}}
\newcommand{\figfont}{\footnotesize}

\title {\large Single-Primer Polymerase Chain Reactions }
\usepackage{authblk}
\author[1]{Ali Atiia}
%\author[2]{Silvia Vidal}
%\author[4]{François Major}
\author[1]{Jérôme Waldispühl}
\renewcommand\Authfont{\fontsize{9}{14}\selectfont}
\affil[1]{School of Computer Science, McGill University, Montreal, Canada}
%\affil[2]{Research Centre on Complex Traits, McGill University, Montreal, Canada}
%\affil[3]{National Institute of Informatics, Tokyo, Japan }
%\affil[4]{Institute for Research in Immunology and Cancer, Montreal, Canada}
\setcounter{Maxaffil}{0}
\renewcommand\Affilfont{\fontsize{7}{14}\selectfont} % the second number determines spacing between lines
%=================================================================================
\date{}
%end_custom_header
%begin_custom_content
\begin{document}
	\maketitle
	\begin{abstract}
			
		% \tiny, \scriptsize, \footnotesize, \small, \normalsize, \large, \Large, \LARGE, \huge, and \Huge.
		
		 Polymerase chain reaction (PCR) is an indispensable molecular biology technique
		 used to selectively amplify a segment of DNA using a thermostable polymerase that extends two short oligonucleotides
		 that complement the 3' ends of the target sequence. The annealing of the two primers
		 to the target extremities at the same temperature is the most important factor in determining the success of the reaction,
		 as misannealing of either or both to non-target regions (or to themselves, "primer-dimer") can lead to erroneous 
		 ampilicons. Having different annealing temperatures can also hinder the success of the reaction, requiring the setting of the reaction
		 annealing temperature to the lowest of the two which can also lead to various problems. Primer design has therefore emerged
		 as one of the most important steps to ensure successful reactions with high yields. There exist many bioinformatic tools
		 to help achieve that, but in practice some primer pairs may still not work well together. Furthermore, when the GC content of 
		 the two extremeties are greatly different, it may become a difficult task to find a suitable primer pair, especially when
		 the choice of primer locations is limited. For example, one primer at one extremity may be a T7 promoter while the other 
		 is part of the coding sequence at another. Here we present a simple technique of inserting the complement of one primer into 
		 one extremity, thereby allowing the use of one primer to anneal to both extremeties of the target ampilicon, thereby 
		 eliminating the need to optimize the design or reaction conditions that suit two different primers. The involved procedure
		 employ standard molecular biology procedure (phosphorylation, digestion, and ligation) and does not require
		 any special sequence modification beside the insertion of one sequence. The method can simplify the engineering 
		 and mass production of plasmids and custom in-vitro transcription templates (for sure in gene (protein) design for example).
		 Designing the 2-way primer to have an annealing temperature equals to that of the extension also eliminates the need
		 for an annealing step during the cycling program of a reaction, resulting in a significant reduction of total reaction time.  
		 	
		\begin{comment}	
			applications:
				de novo gene (protein) design 
				plasmid preparations
			
			the diagnosis and monitoring of genetic diseases, identification of criminals (in the field of forensics), and studying the 
			function of a targeted segment of DNA.
		
			a variety of applications.[4][5] These include DNA cloning for sequencing, DNA-based phylogeny, or functional analysis of 
			genes; the diagnosis of hereditary diseases; the identification of genetic fingerprints (used in forensic sciences and DNA
			paternity testing); and the detection of pathogens in nucleic acid tests for the diagnosis of infectious diseases.
 		\end{comment}
	\end{abstract}
%end_custom_content
\printbibliography
\end{document}