%begin_custom_header
\documentclass[11pt]{article}	% RECOMB: "at least 11 point font size on U.S. standard 8 1/2 by 11 inch paper with no less than one inch margin all around."				
\usepackage[utf8]{inputenc}   % umlauts etc.
\usepackage[english]{babel}
\usepackage [autostyle, english = american]{csquotes}
\MakeOuterQuote{"}
\usepackage{hyperref}
\usepackage{array}
% ----------------------------------
\usepackage[backend=biber,style=nature,sorting=none,url=false]{biblatex}
% url = false. There are also isbn, doi etc., similar options. 
\addbibresource{/Users/mohammedalshamrani/Downloads/School/Waldispul/Publishing/z-misc/zotero-library/my_library.bib}
% ----------------------------------
% Citation style 	biblatex stylename
% ----------------------------------
% 	ACS				chem-acs
% 	AIP				phys (*)
% 	Natur			nature
% 	Science			science
% 	IEEE			ieee
% 	Chicago			chicago-authordate
% 	MLA				mla
% 	APA				apa
% ----------------------------------
% sorting options:
% ----------------------------------
%	nty 		Sort by name, title, year.
%	nyt 		Sort by name, year, title.
%	nyvt 		Sort by name, year, volume, title.
%	anyt 		Sort by alphabetic label, name, year, title.
%	anyvt 		Sort by alphabetic label, name, year, volume, title.
%	ynt 		Sort by year, name, title.
%	ydnt 		Sort by year (descending), name, title.
%	none 		Do not sort at all. All entries are processed in citation order.
% ----------------------------------
\newcommand{\harpoon}{\overset{\rightharpoonup}}
\newtheorem{theorem}{Theorem}
\usepackage{verbatim} % multiline comment
\usepackage{graphicx}
\graphicspath{{/Users/mohammedalshamrani/Downloads/School/Waldispul/Publishing/Paper_04/fig/}}
\setlength\fboxsep{0pt} % figure border padding
\setlength\fboxrule{1pt} % figure outline
\usepackage[fleqn]{amsmath}  % also \documentclass[fleqn]{article}
\usepackage[margin=1in]{geometry}
\abovedisplayskip=0pt
\abovedisplayshortskip=0pt
\belowdisplayskip=0pt
\belowdisplayshortskip=0pt
\setlength{\mathindent}{0pt}
\usepackage{amsfonts} % for R (real numbers)
\usepackage{float}
\usepackage[font=scriptsize,labelfont=bf]{caption}

\usepackage[percent]{overpic}
\usepackage[export]{adjustbox}
% ----------------------------------
%Squeezing the Vertical White Space
%http://www.terminally-incoherent.com/blog/2007/09/19/latex-squeezing-the-vertical-white-space/
% 	THIS FIXES THE PROBLEM OF SUBSECTIONS STARTING IN A NEW PAGE
\setlength{\parskip}{0pt}
\setlength{\parsep}{10pt}
\setlength{\headsep}{0pt}
\setlength{\topskip}{0pt}
\setlength{\topmargin}{0pt}
\setlength{\topsep}{0pt}
\setlength{\partopsep}{10pt}
\usepackage[compact]{titlesec}
\titlespacing{\section}{0pt}{*2}{*2} % {left margin} {above-skip} {below-kip} , The * notation replaces the formal notation using plus/minus and etc. 
\titlespacing{\subsection}{0pt}{*1}{*1}
\titlespacing{\subsubsection}{0pt}{*1}{*1}
% ----------------------------------
\newenvironment{absolutelynopagebreak}
  {\par\nobreak\vfil\penalty0\vfilneg
   \vtop\bgroup}
  {\par\xdef\tpd{\the\prevdepth}\egroup
   \prevdepth=\tpd}
% ----------------------------------
\newcommand{\bfl}{\begin{flushleft}}
\newcommand{\efl}{\end{flushleft}}
\newcommand{\mys }{\hspace{0.1cm}}
\newcommand{\figfont}{\footnotesize}


\usepackage[table]{xcolor} % for \arrayrulecolor{yellow}, changes hline and vline colors
\usepackage{geometry} %
\usepackage{array}
%begin_custom_header
\usepackage{lmodern}
\usepackage{multirow, booktabs} % http://tex.stackexchange.com/questions/328793/table-custom-cell-vertical-alignment					
%end_custom_header
\begin{document}
%begin_custom_content
\section{Method} \label{sec:method}
		%\parbox[H]{\textwidth}{
			DNA templates were constructed using ligation from smaller synthesized oligonucleotides (Biocorp). Segments of each 
			node/edge sequence were concatenated using splint ligation with T4 DNA ligase. The following example illustrates the 
			assembly and ligation of N0’s DNA template: 
			%\newline
			
			% insert figure
		
			The splint oligonucleotide (black, underlined) and the constituent oligonucleotides of a node (red, blue, and green) 
			were mixed at 10 uM concentration each in a 30-ul ligation reaction (50 mM Tris-HCl, 10 mM MgCl2, 1 mM ATP, 10 mM DTT. 
			13u/ul T4 DNA ligase (NEB)) for 1 hour at room temperature. Donor oligonucleotides (blue and green) are phosphorylated 
			with T4 Polynucleotide Kinase (PNK) (NEB) prior to ligation, and in the same ligation reaction conditions. PNK adds a 
			phosphorous residue at the 5’ end of donor strands, a prerequisite for ligation. 
			%\newline
		
			The ligation product is used as template in a PCR reaction. 0.5 ul of ligation reaction (5 picomole total concentration) 
			is used as template in a 25-ul PCR reaction using Phusion DNA polymerase (NEB) and carried out for 40 cycles. By design, one 
			oligonucleotide serves as both the forward and backward primer (since the primer sequence, red in this example, is purposefully 
			ligated 5’ of the template). This eliminates the need to optimize melting temperature condition to satisfy two primers. 
			The following illustration shows the resulting dsDNA template from PCR for N0, with primer sequence shown in bold orange, 
			and the T7 promoter region underlined:  
			%\newline

			% insert figure

			PCR products are used as template in in-vitro transcription (IVT) reactions at a concentration of 20 ng/ul in 
			total reaction volume of 50ul (40 mM Tris-HCl, 6 mM MgCl2, 1.5 mM DTT, 2 mM spermidine, 1U/ul T7 (NEB)). The reaction is 
			carried out for 2-4 hours at 37 degrees Celsius and subsequently treated with 5 units of DNAse I (NEB). Transcription 
			reactions contained 2mM concentration of each NTP, except for N0 where GTP was added to 0.5mM concentration while m7G 
			analog (NEB) was added to 4 mM concentration (to facilitate ribosomal translation of transcripts beginning with N0). 
			In IVT reactions of N1 to N6, guanosine monophosphate (GMP) (Sigma) was added to a 2mM concentration while guanosine 
			triphosphate (GTP) was added to a 0.5mM concentration (to facilitate RNA ligation, since ligase requires monophosphate 
			at the 5’ donor RNA). The example below shows N0’s IVT, with the arrow indicating the transcription start site of T7 polymerase. 
			In all templates, the 1st transcribed base is G (preferred by T7) and the 2nd/3rd are CG when possible, as this has been shown 
			to further improve transcription yield [26]: 
			%\newline
		
			% insert figure

			(4)	N6 RNA sequence is polyadenylated using E. coli. poly(A) polymerase (NEB) in a total reaction volume of 10ul at 
			concentration of 5 ng/ul (50 mM Tris-HCl 250 mM NaCl 10 mM MgCl2, 0.5U/ul poly(A)), in order to facilitate ribosomal 
			translation of sequences ending with N6 since the ribosomal translation mix to be used is from eukaryotes (Promega’s 
			Human In Vitro Translation system) and polyadenation is a prerequisite for mRNA stability and successful translation [27]. 
			%\newline
			
			The RNA transcripts are ligated using T4 RNA Ligase 2 (NEB) at a concentration of 10uM each transcript (nodes and edges) 
			in a total reaction volume of 30ul (50 mM Tris-HCl 10 mM MgCl2 2 mM DTT, 1U/ul T4 Ligase). 
			%\newline
			
			The solution to A-HPP is an mRNA sequence encoding for the enhanced fluorescent green protein (EGFP). The translation 
			step has not yet been implemented. The anatomy of the ligation product encoding for the correct A-HPP solution is shown 
			below (consecutive node sequences shown in different colors, underlined sequence = ribosome binding site (RBS); AUG = start codon, 
			which is part of N0, UAA=stop codon, which is part of N6; lower-case sequence at the 3’ = polyadenylation of N6):

		
		
%end_custom_content
\printbibliography
\end{document}