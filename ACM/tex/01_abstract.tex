\begin{abstract}
	Virtually all molecular interactions networks, independent of organism and physiological context, have majority-leaves minority-hubs (mLmH) topology. 
	Current generative models of this topology are based on controversial 
	hypotheses that, controversy aside, demonstrate sufficient but not necessary evolutionary conditions for its emergence.
	Here we show that the circumvention of computational intractability provides sufficient and (assuming P$\neq$NP) 
	necessary conditions for the emergence of the mLmH property. 
	Evolutionary pressure on molecular interaction networks is simulated by randomly labelling some interactions as `beneficial' and others `detrimental'. Each gene is 
	subsequently given a benefit (damage) score according to how many beneficial (detrimental) interactions it is projecting onto or attracting from other genes. 
	The problem of identifying which subset of genes should ideally be conserved and which deleted, so as to maximize (minimize)  
	the total number of beneficial (detrimental) interactions network-wide, is NP-hard. 
	An evolutionary algorithm that
	simulates hypothetical instances of this problem and selects for networks that produce the easiest instances leads
	to networks that possess the mLmH property. The degree distributions of synthetically evolved networks match those of publicly available  experimentally-validated biological networks from many phylogenetically-distant organisms. 
\end{abstract}

\begin{comment}
		%The debate over what the structural properties of biological networks (BNs) are,
		%how they have emerged, and why they should be considered an evolutionary adaptation, 
		%has been ongoing for almost two decades.  
		%
		%The evolutionary advantage of this topology and the universal law(s) that necessitated its emergence is still unknown.
        %
		Virtually all molecular interactions networks, independent of organism and physiological context, have majority-leaves minority-hubs (mLmH) topology. 
		The evolutionary advantage of this topology and the universal law(s) that necessitated its emergence is not conclusively known.  
		%
		%There is ongoing debate as what the evolutionary forces that have moulded 
		%BNs into their distinct majority-leaves minority-hubs (mLmH) topology are.
		Existing hypotheses %, making different assumptions and operating at different levels of abstraction,
		demonstrate sufficient but not necessary conditions for the emergence of mLmH and therefore one can rule
		out the plausibility of the other.
		%Nor can any existing model rule out the hypothesis that mLmH is a mere byproduct of non-adaptive
		%evolutionary forces such as genetic drift and mutation. 
		We approached the issue from a computational complexity perspective.
		Evolutionary pressure on a BN is simulated by randomly labelling some interactions "beneficial" and others "detrimental". Each gene is 
		subsequently given a benefit (damage) score according to how many beneficial (detrimental) interactions it is projecting onto or attracting from other genes. 
		The problem of identifying what subset of genes should ideally be conserved and which deleted, so as to maximize (minimize)  
		the total number of beneficial (detrimental) interactions under a simulated evolutionary pressure, is NP-hard. An evolutionary algorithm
		that simulating hypothetical instances of this problem and selecting for easy instances produces, after 4000 generations, 
		networks of indistinguishable topology as that of experimentally-validated real BNs in many organisms. 
		Our results show that optimizing against computational intractability generates mLmH. 
		The circumvention of computational intractability provides sufficient and (assuming P$\neq$NP) 
		necessary conditions for the emergence of mLmH property. 
\end{comment}
\begin{comment}
		Competing hypothesis make different assumptions 
		Arguments around the \textit{what} revolve around the statistical coherence and network data representation of one study versus another. 
		Competing hypothesis as to \textit{how} BNs evolved into their majority-leaves minority-hubs (mLmH) topology 

		what: scale-free or not, small-world or not, essential=connected or not\cite{hahn_molecular_2004}. 
		how:  modularity\cite{clune_evolutionary_2013}, duplication and divergence


\end{comment}
\begin{comment}			
			A hub gene can single-handedly contribute a large number of beneficial interactions while the numerous leaf (small degree) genes, which are more
            likely to be either beneficial or detrimental but not both,
            need not be considered in the (computationally costly) optimization search.


		A complexity-theoretic approach to studying biological networks is proposed. 
		A simple graph representation is used where molecules (DNA, RNA, proteins and chemicals) 
		are vertices and relations between them are directed and signed (promotional (+) or inhibitory (-)) edges. 
		Based on this model, the network evolution problem (NEP) is defined formally as an optimization 
		problem and subsequently proven to be fundamentally hard (NP-hard) by means of reduction from 
		the Knapsack problem (KP). For empirical validation, various biological networks of 
		experimentally-validated interactions are compared against randomly generated networks 
		with varying degree distributions. An NEP instance is created using a given real or 
		synthetic (random) network. After being reverse-reduced to a KP instance, each NEP instance 
		is fed to a KP solver and the average achieved knapsack value-to-weight ratio is recorded 
		from multiple rounds of simulated evolutionary pressure. The results show that biological 
		networks (and synthetic networks of similar degree distribution) achieve the highest 
		ratios at maximal evolutionary pressure and minimal error tolerance conditions. 
		The more distant (in degree distribution) a synthetic network is from biological 
		networks the lower its achieved ratio. The results shed light on how computational 
		intractability has shaped the evolution of biological networks into their current topology.

\end{comment}